\section{Amplifier Linearity and Intermodulation (Lab 2)}
\label{sec:lab2}

\subsection*{Exercise 2A: Second--order intermodulation products}

A weakly nonlinear device may be written as 
\( y = a_1x + a_2x^2 + a_3x^3 + \dots \).  
For a two--tone input \( x = A(\cos\omega_1 t + \cos\omega_2 t) \), the
\(a_2x^2\) term generates components at \(f_1 \pm f_2\). The fundamental power
scales as \(A^2\) while IM2 power scales as \(A^4\), giving a 2:1 slope on a dB
input–output plot.

The IM2 extrapolation is \(P_{\mathrm{IM2}} = 2P_{\mathrm{in}} + C_2\).
At the intercept point, \(IP2 = 2IP2 + C_2\), hence \(C_2 = -IP2\).  
Expressing the IM2 level in terms of the fundamental output \(P_0\) gives
\[
P_{\mathrm{IM2}} = 2P_0 - IP2.
\]

\subsection*{Exercise 2B: Third--order intermodulation products}

The cubic term produces the IM3 tones at \(2f_1 \pm f_2\) and \(2f_2 \pm f_1\).  
Here the IM3 amplitude scales as \(A^3\), so IM3 power scales as \(A^6\),
resulting in a 3:1 slope.

Using \(P_{\mathrm{IM3}} = 3P_{\mathrm{in}} + C_3\) and the intercept condition
\(IP3 = 3IP3 + C_3\) gives \(C_3 = -2IP3\).  
Hence,
\[
P_{\mathrm{IM3}} = 3P_0 - 2IP3.
\]

\subsection*{Exercise 2C: Two--tone distortion analysis}

Two tones at \(f_1 = 900\,\mathrm{MHz}\) and \(f_2 = 950\,\mathrm{MHz}\), each at
\(-10\,\mathrm{dBm}\), drive an amplifier with  
\(G=20\,\mathrm{dB}\), \(IP2 = 25\,\mathrm{dBm}\), \(IP3 = 30\,\mathrm{dBm}\).
The fundamentals at the output are
\[
P_0 = -10 + 20 = 10~\mathrm{dBm}.
\]

\textit{Second--order products:}  
Frequencies: \(f_2 - f_1 = 50\,\mathrm{MHz}\) and \(f_1 + f_2 = 1850\,\mathrm{MHz}\).  
Power:
\[
P_{\mathrm{IM2}} = 2P_0 - IP2 = 20 - 25 = -5~\mathrm{dBm}.
\]

\textit{Third--order products:}  
Frequencies: \(2f_1 - f_2 = 850\,\mathrm{MHz}\), \(2f_2 - f_1 = 1000\,\mathrm{MHz}\).  
Power:
\[
P_{\mathrm{IM3}} = 3P_0 - 2IP3 = 30 - 60 = -30~\mathrm{dBm}.
\]

\begin{table}[H]
\centering
\small
\begin{tabular}{l c c}
\toprule
Product & Frequency (MHz) & Power (dBm) \\
\midrule
Fundamentals & 900, 950 & 10 \\
IM2 (diff / sum) & 50, 1850 & -5 \\
Harmonics & 1800, 1900 & -11 \\
IM3 (lower / upper) & 850, 1000 & -30 \\
Higher IM3 & 2700--2850 & -30 to -39 \\
\bottomrule
\end{tabular}
\caption{Calculated IM2 and IM3 distortion products.}
\end{table}

\paragraph{Comparison with ADS simulation.}
The ADS spectrum shows IM2 tones at \(-5\,\mathrm{dBm}\) and IM3 tones clustered
around \(-30\,\mathrm{dBm}\), matching theoretical predictions. Higher--frequency
IM3 components appear weaker due to amplifier roll--off.

\paragraph{Practical significance.}
IM2 products generally fall far from the wanted band and can be filtered,
whereas IM3 products lie close to the desired signals and therefore dominate
linearity requirements, adjacent--channel interference and blocking behaviour.
This is why IP3 is the primary figure of merit for RF amplifier linearity.

\begin{figure}[H]
\centering
\includegraphics[width=0.9\textwidth]{figures/Lab 2/Exercise 1c/Pout.png}
\caption{ADS output spectrum for the two--tone test, showing fundamental, IM2
and IM3 components.}
\label{fig:q1c_spectrum}
\end{figure}

\subsection*{Exercise B (Q2): LNA two--tone IM3 products}

A low-noise amplifier (LNA) with gain \(G = 30\)~dB and input third--order
intercept point \(IIP3 = +5\)~dBm is driven by two equal tones at
\(f_1 = 1.9\)~GHz and \(f_2 = 2.1\)~GHz, each with input power
\(P_{\mathrm{in}} = -10\)~dBm. The corresponding output intercept point is
\[
OIP3 = IIP3 + G = 5 + 30 = 35~\text{dBm}.
\]
The amplified fundamentals at the output are
\[
P_0 = P_{\mathrm{in}} + G = -10 + 30 = 20~\text{dBm}.
\]

\paragraph{IM3 frequencies.}
Third--order intermodulation products occur at the near-band frequencies
\[
2f_1 - f_2 = 1.7~\text{GHz}, \qquad 2f_2 - f_1 = 2.3~\text{GHz},
\]
with additional high-frequency components at \(2f_1 + f_2 = 5.9\)~GHz and
\(2f_2 + f_1 = 6.1\)~GHz.

\paragraph{IM3 power levels.}
Using the output-referred relation
\[
P_{\mathrm{IM3}} = 3P_0 - 2OIP3,
\]
gives
\[
P_{\mathrm{IM3}} = 3(20) - 2(35) = -10~\text{dBm}.
\]
Thus each IM3 product is predicted to appear at about \(-10\)~dBm at the LNA
output.

\paragraph{Comparison with ADS simulation.}
The ADS spectrum in Fig.~\ref{fig:lab2_q2_sim} shows IM3 tones at 1.7 and
2.3~GHz close to the predicted level, with any residual discrepancy attributable
to device roll-off and filtering around the 100~MHz receiver bandwidth.

\begin{figure}[H]
  \centering
  \includegraphics[width=0.6\textwidth]{figures/Lab 2/Exercise 2/SimResults.PNG}
  \caption{ADS two--tone spectrum for Exercise~B (Q2).}
  \label{fig:lab2_q2_sim}
\end{figure}

\subsection*{Exercise C (Q3): Cascaded receiver linearity}

The receiver chain consists of an RF filter, amplifier (Amp1), mixer and IF
filter. The gains, noise figures and IIP3 values are given in the table in the
question. First, output intercept points are found using
\(OIP3_i = IIP3_i + G_i\):
\[
\begin{array}{l c c}
\text{Stage} & G_i~(\text{dB}) & OIP3_i~(\text{dBm}) \\ \hline
\text{RF filter} & -1 & 98 \\
\text{Amp1} & 20 & 10 \\
\text{Mixer} & -6 & 0 \\
\text{IF filter} & -1 & 98 \\
\end{array}
\]
The corresponding linear gains are
\[
G_1 = 0.79,\quad G_2 = 100,\quad G_3 = 0.25,\quad G_4 = 0.79,
\]
giving a total gain of
\[
G_{\mathrm{tot}} = -1 + 20 - 6 - 1 = 12~\text{dB}.
\]

\paragraph{Cascaded OIP3.}
Using the output-referred cascade formula
\[
\frac{1}{OIP3_{\mathrm{cas}}}
= \frac{1}{G_2G_3G_4\,OIP3_1}
 \frac{1}{G_3G_4\,OIP3_2}
 \frac{1}{G_4\,OIP3_3}
 \frac{1}{OIP3_4},
\]
with \(OIP3_1=OIP3_4\approx 6.31\times10^9\)~mW,
\(OIP3_2=10\)~mW and \(OIP3_3=1\)~mW, yields
\[
OIP3_{\mathrm{cas}} \approx 0.564~\text{mW} \;\Rightarrow\;
OIP3_{\mathrm{cas}} \approx -2.47~\text{dBm}.
\]

With two input tones at \(f_1 = 1.6\)~GHz and \(f_2 = 2.0\)~GHz, each at
\(-45\)~dBm, the output fundamentals are
\[
P_0 = -45 + 12 = -33~\text{dBm}.
\]
The predicted IM3 level is therefore
\[
P_{\mathrm{IM3}}
= 3P_0 - 2OIP3_{\mathrm{cas}}
= 3(-33) - 2(-2.47)
= -94.1~\text{dBm}.
\]

\paragraph{IM3 frequencies and practical significance.}
The third--order products occur at
\[
2f_1 - f_2 = 1.2~\text{GHz}, \qquad 2f_2 - f_1 = 2.4~\text{GHz}.
\]
Although these fall outside the 1.6--2.0~GHz IF band and are attenuated by the
IF filter, they indicate the receiver's sensitivity to strong adjacent tones.
The cascaded OIP3 is dominated by the mixer and Amp1 stages, so improving either
stage yields the largest system-level linearity benefit.

\paragraph{Comparison with ADS simulation.}
The ADS spectrum in Fig.~\ref{fig:lab2_q3_sim} shows IM3 products at the same
frequencies and close to the predicted level, confirming the cascade model.

\begin{figure}[H]
  \centering
  \includegraphics[width=0.6\textwidth]{figures/Lab 2/Exercise 3/SimResults.PNG}
  \caption{ADS two--tone spectrum for Exercise~C (Q3).}
  \label{fig:lab2_q3_sim}
\end{figure}
