% Lab 1 content: Link Budget and Propagation

This laboratory introduces the fundamentals of link budgeting by examining how
transmitted power, antenna gains and propagation conditions determine the
strength of a received signal. By modelling a complete transmitter--channel--
receiver chain in ADS, the exercise demonstrates how analytical predictions can
be compared directly with simulated results, establishing a clear understanding
of how frequency, distance and system parameters influence overall link
performance. These concepts form the foundation for later laboratories, where
distortion, filtering, noise and wider RF system behaviour are investigated.

\vspace{2mm}

The theoretical background draws on the Friis transmission relationship and the
free--space path loss model, which describe how power decays with distance and
frequency. Received power is obtained by combining transmitter power, antenna
gains and propagation loss in decibel form. Antenna directivity is central,
with higher frequency systems relying on significant antenna gain to counter
increased path loss. Free--space propagation assumes an unobstructed
line--of--sight link, providing a controlled baseline for analysis. ADS then
allows this ideal behaviour to be compared against scenarios that include
additional losses such as atmospheric absorption or building penetration.

Before validating the system performance using ADS, each scenario was first
evaluated using the link-budget equations provided in the Lab~1
handbook~\cite{Lab1_handbook}. The received power is obtained using:
\[
P_r(dBm) = P_t(dBm) + G_t(dBi) + G_r(dBi)
           + 20\log_{10}\!\left(\frac{\lambda}{4\pi R}\right),
\]
with antenna gains and parabolic reflector diameters calculated using:

\[
G_x(dBi) = 10\log_{10}\!\left(
\eta\frac{\pi^2 d^2}{\lambda^2}
\right).
\]
These expressions provide a baseline prediction for each wireless link prior to
introducing simulation tools or additional channel impairments.

\textbf{Exercise A: 700\,MHz Terrestrial Broadcast Link}\\
For a carrier frequency of $700$\,MHz, the wavelength is:
\[
\lambda = \frac{c}{f} = \frac{3\times10^8}{700\times10^6} = 0.43\;\mathrm{m}.
\]
The transmit power of $5$\,kW corresponds to
\[
P_t = 10\log_{10}(5\times10^6) = 66.99\;\mathrm{dBm}.
\]
The free-space loss over $25$\,km is
\[
\mathrm{FSPL}
= 20\log_{10}\!\left(\frac{0.43}{4\pi\cdot25\times10^3}\right)
= -117.30\;\mathrm{dB}.
\]
The received power is therefore
\[
P_r = 66.99 + 3 + 10 - 117.30
    = -37.31\;\mathrm{dBm}.
\]

\begin{figure}[H]
  \centering
  \begin{subfigure}[b]{0.46\textwidth}
    \centering
    \includegraphics[width=0.8\textwidth]{figures/Lab 1/Exercise A/SimCircuit.PNG}
    \caption{ADS link-budget circuit for Lab~1 Exercise~A.}
    \label{fig:lab1_exA_circuit}
  \end{subfigure}
  \hfill
  \begin{subfigure}[b]{0.46\textwidth}
    \centering
    \includegraphics[width=0.7\textwidth]{figures/Lab 1/Exercise A/SimResults.PNG}
    \caption{Simulated received-power results for Lab~1 Exercise~A.}
    \label{fig:lab1_exA_results}
  \end{subfigure}
  \caption{ADS schematic and simulation results for the Lab~1 link-budget scenario.}
  \label{fig:lab1_circuit_and_sim}
\end{figure}

\textbf{Exercise B: 38\,GHz Backbone Link.}\\

In this exercise, the objective is to determine the diameter of the transmit
parabolic antenna required to achieve a received power of $-50$\,dBm. The
carrier frequency is $38$\,GHz, giving a wavelength of
\[
\lambda = \frac{c}{f} =  \frac{3\times 10^8}{38\times10^9}
        = 7.89\times 10^{-3}\;\mathrm{m}.
\]

The receiver uses a $15$\,cm dish with efficiency $\eta = 0.55$. Its gain is
\[
G_r
= 10\log_{10}\!\left(
\eta\frac{\pi^2 d_r^2}{\lambda^2}
\right)
= 10\log_{10}\!\left(
0.55\frac{\pi^2 (0.15)^2}{(7.89\times10^{-3})^2}
\right)
= 32.92\;\mathrm{dBi}.
\]

The free--space loss over $R = 10$\,km is
\[
\mathrm{FSPL}
= 20\log_{10}\!\left(\frac{\lambda}{4\pi R}\right)
= -144.04\;\mathrm{dB}.
\]

Atmospheric absorption adds a further $10$\,dB attenuation. The system transmit
power is $100$\,mW, corresponding to
\[
P_t = 10\log_{10}(100) = 20\;\mathrm{dBm}.
\]

The link budget expression becomes
\[
P_r = P_t + G_t + G_r + \mathrm{FSPL} - 10.
\]

Setting the target received power $P_r = -50$\,dBm and solving for $G_t$ gives
\[
-50
= 20 + G_t + 32.92 - 144.04 - 10,
\]
\[
-50 = -101.12 + G_t,
\]
\[
G_t = 51.12\;\mathrm{dBi}.
\]

The corresponding transmit-antenna diameter follows from the parabolic reflector
gain expression:
\[
G_t = 10\log_{10}\!\left(
\eta\frac{\pi^2 d_t^2}{\lambda^2}
\right),
\]
\[
d_t
= \lambda
  \sqrt{
  \frac{10^{G_t/10}}{\eta\pi^2}
  }.
\]

Substituting $\eta = 0.55$, $\lambda = 7.89\times10^{-3}$\,m and
$G_t = 51.12$\,dBi,
\[
d_t
= 7.89\times10^{-3}
  \sqrt{
  \frac{10^{5.112}}{0.55\pi^2}
  }
  = 1.22\;\mathrm{m}.
\]

Thus, a transmit-antenna diameter of approximately \textbf{1.22\,m} is required
to achieve a received power of $-50$\,dBm under the stated conditions.

\begin{figure}[H]
  \centering
  \begin{subfigure}[b]{0.46\textwidth}
    \centering
    \includegraphics[width=0.8\textwidth]{figures/Lab 1/Exercise B/Circuit.PNG}
    \caption{ADS circuit schematic for Lab~1 Exercise~B.}
    \label{fig:lab1_exB_circuit}
  \end{subfigure}
  \hfill
  \begin{subfigure}[b]{0.46\textwidth}
    \centering
    \includegraphics[width=0.7\textwidth]{figures/Lab 1/Exercise B/Sim.PNG}
    \caption{Simulation results for the 38\,GHz backbone link.}
    \label{fig:lab1_exB_sim}
  \end{subfigure}
  \caption{ADS schematic and simulation results for the 38\,GHz backbone link
  (Exercise~B). The simulated received power of $-50.04$\,dBm matches the
  analytical prediction.}
  \label{fig:lab1_exB_circuit_and_sim}
\end{figure}

Atmospheric attenuation varies with frequency due to absorption by oxygen and
water vapour, producing alternating low–loss transmission windows and high–loss
absorption bands (see Fig.~\ref{fig:atmospheric_attenuation}). The 38\,GHz
region falls within a relatively favourable window, whereas frequencies near
22\,GHz and 60\,GHz exhibit significantly greater loss. \cite{millimetre_microwave} The $10$\,dB attenuation
term used in this exercise therefore represents a single point on a broader
frequency–dependent profile and illustrates the importance of accounting for
atmospheric absorption in long–distance link design.

\begin{figure}[H]
    \centering
    \includegraphics[width=0.4\textwidth]{figures/atmospheric_absorbtion.svg}
    \caption{Atmospheric attenuation vs frequency, showing absorption features
    from oxygen and water vapour~\cite{atmospheric_attenuation}.}
    \label{fig:atmospheric_attenuation}
\end{figure}


\textbf{Exercise C: 1.8\,GHz Cellular Link}\\[4pt]
In exercise C, further attenuation factors such as fading margin and building
penetration loss are introduced to the link budget. For a $1.8$\,GHz carrier:
\[
\lambda = \frac{3\times10^8}{1.8\times10^9}
        = 0.167\;\mathrm{m}.
\]

The transmitter radiates $100$\,W, corresponding to
\[
P_t = 10\log_{10}(100\,\mathrm{W}/1\,\mathrm{mW})
    = 10\log_{10}(10^5)
    = 50\;\mathrm{dBm}.
\]

The free--space path loss over a $5$\,km link is
\[
\mathrm{FSPL}
= 20\log_{10}\!\left(\frac{\lambda}{4\pi R}\right)
= 20\log_{10}\!\left(
\frac{0.167}{4\pi \cdot 5000}
\right)
= -111.53\;\mathrm{dB}.
\]

The transmitter antenna gain is $G_t = 6$\,dB and the receiver antenna gain is
$G_r = 0$\,dB. Fading margin ($20$\,dB) and building penetration loss ($6$\,dB)
combine to give a total additional attenuation of
\[
L_{\mathrm{extra}} = 20 + 6 = 26\;\mathrm{dB}.
\]

The received power is therefore
\[
P_r = P_t + G_t + G_r + \mathrm{FSPL} - L_{\mathrm{extra}},
\]
\[
P_r = 50 + 6 + 0 - 111.53 - 26,
\]
\[
P_r = -81.53\;\mathrm{dBm}.
\]

\begin{figure}[H]
  \centering
  \begin{subfigure}[b]{0.46\textwidth}
    \centering
    \includegraphics[width=0.8\textwidth]{figures/Lab 1/Exercise C/Circuit.PNG}
    \caption{ADS circuit schematic for Exercise~C.}
    \label{fig:lab1_exC_circuit}
  \end{subfigure}
  \hfill
  \begin{subfigure}[b]{0.46\textwidth}
    \centering
    \includegraphics[width=0.7\textwidth]{figures/Lab 1/Exercise C/Sim.PNG}
    \caption{Simulation results for the 1.8\,GHz link.}
    \label{fig:lab1_exC_sim}
  \end{subfigure}
  \caption{
  ADS schematic and simulation results for the 1.8\,GHz cellular link
  (Exercise~C). The simulated value of $P_r = -81.53$\,dBm matches the analytic
  calculation.
  }
  \label{fig:lab1_exC_circuit_and_sim}
\end{figure}

\paragraph{Exercise D: GEO Satellite Downlink.}
For a $12$\,GHz carrier,
\[
\lambda = \frac{3\times10^8}{12\times10^9} = 0.025\;\mathrm{m}.
\]
The spacecraft and ground-station dish gains follow from
\[
G = 10\log_{10}\!\left(
0.6\frac{\pi^2 d^2}{\lambda^2}
\right),
\]
using diameters of $2$\,m and $0.6$\,m respectively.  
The free-space loss over $40{,}000$\,km is added along with atmospheric
absorption ($2$\,dB), low-noise-amplifier gain ($30$\,dB) and mixer conversion
loss ($10$\,dB):
\[
P_{IF} = P_t + G_t + G_r + \mathrm{FSPL}
         - 2\;\mathrm{dB}
         + 30\;\mathrm{dB}
         - 10\;\mathrm{dB}.
\]
This predicts the IF power prior to validating the chain using ADS.

\begin{figure}[H]
  \centering
  \begin{subfigure}[b]{0.46\textwidth}
    \centering
    \includegraphics[width=0.8\textwidth]{figures/Lab 1/Exercise D/Circuit.PNG}
    \caption{ADS circuit schematic for Lab~1 Exercise~D.}
    \label{fig:lab1_exD_circuit}
  \end{subfigure}
  \hfill
  \begin{subfigure}[b]{0.46\textwidth}
    \centering
    \includegraphics[width=0.7\textwidth]{figures/Lab 1/Exercise D/Sim.PNG}
    \caption{Simulation results for the GEO satellite downlink.}
    \label{fig:lab1_exD_sim}
  \end{subfigure}
  \caption{ADS schematic and simulation results for the GEO satellite downlink (Exercise~D).}
  \label{fig:lab1_exD_circuit_and_sim}
\end{figure}
