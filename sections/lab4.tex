% Lab 4 content: Transistor Characteristics
\section{Transistor Characteristics (Lab 4)}
\label{sec:lab4}

\subsection*{Exercise 1: DC bias and small-signal parameters}

The BJT is modelled with \(V_{BE}=0.7\)~V and \(\beta=100\). For each bias
configuration, Kirchhoff’s voltage law gives the base current, from which
\(I_C=\beta I_B\), \(I_E=I_C+I_B\) and \(V_{CE}\) follow. ADS DC operating-point
results are summarised in Table~\ref{tab:lab4_ex1_dc}.

\begin{table}[H]
  \centering
  \small
  \begin{tabular}{l c c c c c c}
    \toprule
    Circuit & \(I_B\) (A) & \(I_C\) (A) & \(I_E\) (A) & \(V_{BE}\) (V) & \(V_{CE}\) (V) & \(\beta\) \\
    \midrule
    (a) & \(7.94\times10^{-5}\) & \(8.0\times10^{-3}\) & \(8.0\times10^{-3}\) & 0.822 & 4.056 & 100 \\
    (b) & \(2.28\times10^{-5}\) & \(2.0\times10^{-3}\) & \(2.0\times10^{-3}\) & 0.790 & 7.424 & 100 \\
    \bottomrule
  \end{tabular}
  \caption{ADS DC operating-point results for Exercise~1 circuits (a) and (b).}
  \label{tab:lab4_ex1_dc}
\end{table}

For circuit (a) the base current is \(I_B\approx7.94\times10^{-5}\)~A, giving
\(I_C\approx8\)~mA and \(V_{CE}\approx4.06\)~V. Circuit (b) introduces emitter
degeneration, reducing the bias currents to \(I_B\approx2.28\times10^{-5}\)~A,
\(I_C\approx2\)~mA and increasing the collector–emitter voltage to
\(V_{CE}\approx7.42\)~V. The simulated values are consistent with the hand
calculations and show how emitter resistance improves bias stability at the cost
of gain.

\subsection*{Exercise 2: Output characteristics}

Using the Agilent HBT model, \(V_{CE}\) was swept from 0 to 5~V while stepping
\(I_B\) from 20 to 200~\(\mu\)A. The resulting \(I_C\)–\(V_{CE}\) family is shown
in Fig.~\ref{fig:lab4_ex2_results}.

\begin{figure}[H]
  \centering
  \includegraphics[width=0.5\textwidth]{figures/Lab 4/Exercise 2/SimResults.PNG}
  \caption{Collector current versus \(V_{CE}\) for stepped base currents.}
  \label{fig:lab4_ex2_results}
\end{figure}

Each curve exhibits a low-\(V_{CE}\) saturation region followed by a nearly flat
active region where \(I_C\) is set by \(I_B\). The slight upward slope in the
active region indicates finite output resistance (Early effect), so \(I_C\)
increases modestly with \(V_{CE}\). Higher \(I_B\) shifts the curves upward
approximately linearly, matching the expected proportionality between \(I_C\)
and base drive.

\subsection*{Exercise 3: Frequency-dependent gain}

The NPN device was replaced by the HBT model and the small-signal voltage gain
was evaluated from 1 to 10~GHz. Fig.~\ref{fig:lab4_ex3_results} shows the ADS
AC results and derived parameters.

\begin{figure}[H]
  \centering
  \includegraphics[width=1\textwidth]{figures/Lab 4/Exercise 3/SimResults_graph.PNG}
  \caption{ADS AC gain magnitude/phase and extracted bias values.}
  \label{fig:lab4_ex3_results}
\end{figure}

The key values extracted from ADS are listed in Table~\ref{tab:lab4_ex3_vals}.

\begin{table}[H]
  \centering
  \small
  \begin{tabular}{c c c c}
    \toprule
    \(f\) (GHz) & \(|A_v|\) & \(\angle A_v\) (deg) & \\
    \midrule
    \multicolumn{4}{l}{\textit{DC operating point}} \\
    \multicolumn{1}{l}{\(I_B = 3.953\times 10^{-5}\,\text{A}\)} &
    \multicolumn{1}{l}{\(I_C = 4.0\times 10^{-3}\,\text{A}\)} &
    \multicolumn{1}{l}{\(V_{CE} = -29.53\,\text{V}\)} &
    \\
    \midrule
    \multicolumn{4}{l}{\textit{AC gain versus frequency}} \\
    1.0 & 26.924 & 155.852 & \\
    2.0 & 18.515 & 124.852 & \\
    3.0 & 13.411 & 110.292 & \\
    4.0 & 10.405 & 101.365 & \\
    5.0 & 8.480  & 94.953  & \\
    6.0 & 7.157  & 89.872  & \\
    7.0 & 6.198  & 85.584  & \\
    8.0 & 5.473  & 81.811  & \\
    9.0 & 4.908  & 78.398  & \\
    10.0 & 4.457 & 75.250  & \\
    \bottomrule
  \end{tabular}
  \caption{ADS-derived DC bias currents/voltage and small-signal gain for Exercise~3.}
  \label{tab:lab4_ex3_vals}
\end{table}

The voltage gain is highest at low frequency (about 27 at 1~GHz) and rolls off
monotonically to roughly 4–5 by 10~GHz. This behaviour reflects the transistor’s
finite transition frequency: parasitic capacitances reduce transconductance and
introduce additional phase shift as frequency increases. The measured DC bias
currents are in line with those from Exercise~1, so the gain reduction is
dominated by device high-frequency limits rather than a change in operating
point.

\FloatBarrier

