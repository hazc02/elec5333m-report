% Lab 3 content: Filter Behaviour
\section{Filter Behaviour (Lab 3)}
\label{sec:lab3}

\subsection*{Exercise 1: Three--pole Chebyshev lowpass filter}

A third--order Chebyshev lowpass filter was synthesised with
\(F_p=1\)~GHz, \(F_s=3.2\)~GHz, \(A_p=0.1\)~dB and \(A_s=20\)~dB for a
50~\(\Omega\) source/load. The initial lumped LC prototype was then transformed
to an equivalent transmission--line network and finally to a microstrip
implementation on the specified substrate. Fig.~\ref{fig:lab3_ex1_circuits}
summarises each topology alongside its simulated response.

\begin{figure}[H]
  \centering
  \begin{subfigure}[b]{0.4\textwidth}
    \centering
    \includegraphics[width=\textwidth]{figures/Lab 3/Exercise 1/LC_Filter.PNG}
    \caption{LC prototype.}
  \end{subfigure}
  \hfill
  \begin{subfigure}[b]{0.59\textwidth}
    \centering
    \includegraphics[width=\textwidth]{figures/Lab 3/Exercise 1/LC_Filter_Sim.PNG}
    \caption{LC response.}
  \end{subfigure}

  \vspace{0.6em}

  \begin{subfigure}[b]{0.4\textwidth}
    \centering
    \includegraphics[width=\textwidth]{figures/Lab 3/Exercise 1/TRL_Circuit.PNG}
    \caption{TLine transformation.}
  \end{subfigure}
  \hfill
  \begin{subfigure}[b]{0.59\textwidth}
    \centering
    \includegraphics[width=\textwidth]{figures/Lab 3/Exercise 1/TRL_Sim.PNG}
    \caption{TLine response.}
  \end{subfigure}

  \vspace{0.6em}

  \begin{subfigure}[b]{0.4\textwidth}
    \centering
    \includegraphics[width=\textwidth]{figures/Lab 3/Exercise 1/MS_Circuit.PNG}
    \caption{Microstrip layout.}
  \end{subfigure}
  \hfill
  \begin{subfigure}[b]{0.59\textwidth}
    \centering
    \includegraphics[width=\textwidth]{figures/Lab 3/Exercise 1/MS_Sim.PNG}
    \caption{Microstrip response.}
  \end{subfigure}
  \caption{Exercise~1 lowpass filter topologies and corresponding simulated
  S-parameters.}
  \label{fig:lab3_ex1_circuits}
\end{figure}

All three implementations meet the 0.1~dB passband ripple near 1~GHz and achieve
at least 20~dB attenuation by 3.2~GHz. The TLine and microstrip versions show
slightly increased loss and a small shift of the stopband edge due to
distributed effects and substrate dispersion, but preserve the intended
Chebyshev shape.

\subsection*{Exercise 2: Microstrip Chebyshev bandpass filter}

A microstrip Chebyshev bandpass filter was designed with
\(F_0=12\)~GHz, \(\Delta f=1.3\)~GHz and 0.1~dB passband ripple, targeting
30~dB rejection between 9 and 15~GHz. Fig.~\ref{fig:lab3_ex2_circuits} shows the
ADS lumped prototype and the resulting microstrip realisation after substrate
parameterisation.

\begin{figure}[H]
  \centering
  \begin{subfigure}[b]{0.48\textwidth}
    \centering
    \includegraphics[width=0.85\textwidth]{figures/Lab 3/Exercise 2/LC_Filter_Circuit.PNG}
    \caption{Lumped prototype.}
  \end{subfigure}
  \hfill
  \begin{subfigure}[b]{0.48\textwidth}
    \centering
    \includegraphics[width=0.85\textwidth]{figures/Lab 3/Exercise 2/MS_Filter_Circuit.PNG}
    \caption{Microstrip realisation.}
  \end{subfigure}
  \caption{Bandpass filter circuits before and after microstrip conversion.}
  \label{fig:lab3_ex2_circuits}
\end{figure}

The return loss and insertion loss of the microstrip bandpass filter are shown
in Fig.~\ref{fig:lab3_ex2_sparams}. The passband is centred close to 12~GHz with
approximately 1.3~GHz 3~dB bandwidth, and \(S_{11}\) remains below \(-16\)~dB
through most of the passband, consistent with the 0.1~dB ripple specification.
Outside the band, the response rolls off rapidly and exceeds 30~dB attenuation
across the 9--15~GHz stop region.

\begin{figure}[H]
  \centering
  \begin{subfigure}[b]{0.49\textwidth}
    \centering
    \includegraphics[width=\textwidth]{figures/Lab 3/Exercise 2/MS_Filter_S11.PNG}
    \caption{Return loss \(S_{11}\).}
  \end{subfigure}
  \hfill
  \begin{subfigure}[b]{0.49\textwidth}
    \centering
    \includegraphics[width=\textwidth]{figures/Lab 3/Exercise 2/MS_Filter_S21.PNG}
    \caption{Insertion loss \(S_{21}\).}
  \end{subfigure}
  \caption{Microstrip bandpass filter S-parameters.}
  \label{fig:lab3_ex2_sparams}
\end{figure}

\FloatBarrier

