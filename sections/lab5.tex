% Lab 5 content: Time-Domain Distortion and Antenna Performance
\section{Time-Domain Distortion and Antenna Performance (Lab 5)}
\label{sec:lab5}

\subsection*{Exercise 1: Time-domain non-linear distortion}

The common-emitter amplifier was analysed in DC/AC to establish its operating
point and small-signal gain, then simulated in transient with a swept input
amplitude. At low drive levels the output is sinusoidal and scales linearly with
the input. As the input amplitude is increased, the transistor leaves its
linear region and the output exhibits clipping/compression, as shown in
Fig.~\ref{fig:lab5_ex1_waveforms}.

\begin{figure}[H]
  \centering
  \includegraphics[width=0.8\textwidth]{figures/Lab 5/Exercise 1/SimResults.PNG}
  \caption{Transient input (red) and output (blue) waveforms for increasing
  input amplitude.}
  \label{fig:lab5_ex1_waveforms}
\end{figure}

This distortion corresponds to the gain becoming signal-dependent. In the
frequency domain a clipped sinusoid requires additional harmonic components to
reconstruct the waveform, so harmonic balance would show rising harmonics and,
for multi-tone inputs, intermodulation products. The onset of visible clipping
therefore marks the practical linearity limit for the stage.

\subsection*{Exercise 2: Rectangular microstrip patch antenna}

Using the FR4 substrate parameters (\(\varepsilon_r=4.6\), \(h=1.6\)~mm) and a
target resonance of \(f_r=2.4\)~GHz, the patch dimensions follow from the design
equations:
\[
W=L=\frac{c}{2f_r\sqrt{\varepsilon_r}} \approx 29.14~\text{mm}, \qquad
H=0.822\,\frac{L}{2} \approx 11.98~\text{mm},
\]
\[
Y=\frac{W}{5}\approx 5.83~\text{mm}, \qquad
X=Z=\frac{2W}{5}\approx 11.66~\text{mm}.
\]
The layout was implemented in ADS Momentum and simulated from 1–4~GHz.

\begin{figure}[H]
  \centering
  \includegraphics[width=0.65\textwidth]{figures/Lab 5/Exercise 2/Antenna.PNG}
  \caption{ADS layout of the 2.4~GHz rectangular patch antenna.}
  \label{fig:lab5_ex2_layout}
\end{figure}

The simulated input match is shown in Fig.~\ref{fig:lab5_ex2_s11}. A clear
resonance occurs close to 2.4~GHz where \(S_{11}\) dips below \(-20\)~dB,
confirming good impedance matching at the design frequency.

\begin{figure}[H]
  \centering
  \includegraphics[width=0.7\textwidth]{figures/Lab 5/Exercise 2/S11Results.PNG}
  \caption{Simulated return loss \(S_{11}\) of the patch antenna.}
  \label{fig:lab5_ex2_s11}
\end{figure}

Far-field results at 2.4~GHz are summarised in Fig.~\ref{fig:lab5_ex2_patterns}.
The antenna radiates a broadside pattern with peak directivity of about
6.3~dBi and realised gain around 4.9~dBi, corresponding to roughly 73\%
efficiency. The polar cuts in the principal planes show the expected
single-lobe behaviour for a rectangular patch and low cross-polarisation in the
main beam.

\begin{figure}[H]
  \centering
  \begin{subfigure}[b]{0.49\textwidth}
    \centering
    \includegraphics[width=\textwidth]{figures/Lab 5/Exercise 2/RadiationPattern.PNG}
    \caption{Gain/directivity and efficiency.}
    \label{fig:lab5_ex2_rad_summary}
  \end{subfigure}
  \hfill
  \begin{subfigure}[b]{0.49\textwidth}
    \centering
    \includegraphics[width=\textwidth]{figures/Lab 5/Exercise 2/FarFieldCut1.PNG}
    \caption{Far-field cut (plane 1).}
    \label{fig:lab5_ex2_cut1}
  \end{subfigure}

  \vspace{0.6em}

  \includegraphics[width=0.7\textwidth]{figures/Lab 5/Exercise 2/FarFieldCut2.PNG}
  \caption{Simulated far-field characteristics of the patch antenna at 2.4~GHz.}
  \label{fig:lab5_ex2_patterns}
\end{figure}

\FloatBarrier
\clearpage
